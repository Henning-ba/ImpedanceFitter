\documentclass[11pt]{beamer}
\usetheme{Warsaw}
\usepackage[utf8]{inputenc}
\usepackage[english]{babel}
\usepackage{amsmath}
\usepackage{amsfonts}
\usepackage{amssymb}
\usepackage{graphicx}
\usepackage{listings}
\usepackage{xcolor}

\definecolor{codegreen}{rgb}{0,0.6,0}
\definecolor{codegray}{rgb}{0.5,0.5,0.5}
\definecolor{codepurple}{rgb}{0.58,0,0.82}
\definecolor{backcolour}{rgb}{0.95,0.95,0.92}

\lstdefinestyle{mystyle}{
    backgroundcolor=\color{backcolour},   
    commentstyle=\color{codegreen},
    keywordstyle=\color{magenta},
    numberstyle=\tiny\color{codegray},
    stringstyle=\color{codepurple},
    basicstyle=\ttfamily\footnotesize,
    breakatwhitespace=false,         
    breaklines=true,                 
    captionpos=b,                    
    keepspaces=true,                 
    numbers=left,                    
    numbersep=5pt,                  
    showspaces=false,                
    showstringspaces=false,
    showtabs=false,                  
    tabsize=2
}

\lstset{style=mystyle}


\author{Julius Zimmermann}
\title{Introduction to ImpedanceFitter}
%\setbeamercovered{transparent} 
%\setbeamertemplate{navigation symbols}{} 
%\logo{} 
%\institute{} 
%\date{} 
%\subject{} 
\begin{document}

\begin{frame}
\titlepage
\end{frame}

%\begin{frame}
%\tableofcontents
%\end{frame}

\begin{frame}{Use case}

\begin{itemize}
\item Multiple impedance spectra have been recorded
\begin{itemize}

\item Visualization 
\item Validity check
\item Fitting, data analysis

\end{itemize}
\item An impedance model is available
\begin{itemize}
\item Expected impedance in a defined frequency range
\item Decomposition of the model 
\item Estimating dielectric properties
\end{itemize}
\end{itemize}

\begin{block}{Advantages}
\begin{itemize}
\item High-level interface, batch processing
\item Flexible choice of fitting algorithm
\item Many relevant models implemented
\end{itemize}

\end{block}

\end{frame}

\begin{frame}{Getting started}
\begin{itemize}
\item Prerequisite: Python $>=$ 3.6 installed, download from \url{https://www.python.org}. \textbf{Make sure that Python is added to the PATH variable on Windows!}
\item Ideally: Jupyter installed (to have Jupyter notebooks) from \url{https://jupyter.org/}
\item Installation is a one-liner:
\begin{center}
\textbf{pip install impedancefitter}
\end{center}
\item Bugs? Please report them to me!
\item Documentation on the website: \url{https://impedancefitter.readthedocs.io}
\item Feedback is highly appreciated!
\end{itemize}

\end{frame}

\begin{frame}[fragile]{General idea}

\begin{itemize}
\item The impedance measured in electrochemical impedance spectroscopy (EIS) is assumed to be the linear response of the system
\item[$\Rightarrow$] Only linear models (equivalent circuits) considered
\item The impedance is modelled by a series or parallel connection of circuit elements and pre-defined circuits
\item Realisation in ImpedanceFitter: model is defined in a string and parsed
\item Example: Randles circuit
\begin{lstlisting}
model = 'R_s + parallel(R_ct + W, C)'
model = 'Randles'
\end{lstlisting}
\end{itemize}

\end{frame}

\begin{frame}[fragile]{The model}

\begin{itemize}
\item The model is connects elements in series by \lstinline{+} and in parallel by \lstinline{parallel(a, b)}
\item Example: Randles circuit
\begin{lstlisting}
model = 'R_s + parallel(R_ct + W, C)'
model = 'Randles'
\end{lstlisting}
\item Question: find out more about the implementation of the Randles circuit and its parameters on the website
\item To compute the impedance: generate an equivalent circuit and evaluate it
\begin{lstlisting}
Z = lmfit_model.eval(omega=2. * numpy.pi * frequencies, ct_R=Rct, s_R=Rs, C=C0, Aw=Aw)
\end{lstlisting}
\item[!] If you are new to Python please don't hesitate to ask questions!
\end{itemize}
\end{frame}

\begin{frame}{Post-processing the model data}

\begin{itemize}
\item Various functions to plot the data are available: \lstinline{'plot_admittance', 'plot_bode', 'plot_cole_cole', 'plot_comparison_dielectric_properties', 'plot_complex_permittivity', 'plot_dielectric_dispersion', 'plot_dielectric_modulus', 'plot_dielectric_properties', 'plot_impedance', 'plot_resistance_capacitance'}
\item[$\Rightarrow$] Documentation on website or through docstrings
\item Every device has a frequency independent unit capacitance $C_0$, which relates the impedance to the dielectric properties
\begin{equation}
Z = \frac{1}{j \omega \hat{\varepsilon} C_0}~\mathrm{with}~ \hat{\varepsilon} = \varepsilon_r - \frac{j \sigma}{\omega \varepsilon_0} \enspace .
\end{equation}
\item[$\Rightarrow$] Function: \lstinline{impedancefitter.utils.return_diel_properties}
\end{itemize}
\end{frame}

\begin{frame}[fragile]{EIS data analysis: loading the data}
\begin{itemize}
\item Available file formats: \url{https://impedancefitter.readthedocs.io/en/latest/examples/fileformats.html}
\item[!] Reach out to me if you want other file formats to be implemented
\item General idea: fixed frequencies in one file, probably multiple impedance spectra (real and imaginary part)
\item Other representations (polar, R-C, etc.) need to be converted before loaded
\item Simplest starting point:
\begin{lstlisting}
fitter = impedancefitter.Fitter('CSV')
fitter.visualize_data()
fitter.visualize_data(allinone=True)
\end{lstlisting}
\end{itemize}
\end{frame}

\begin{frame}{EIS data analysis: validating the data}
\begin{itemize}
\item Impedance data have to fulfil Kramers-Kronig relations
\item Test based on general equivalent circuit\footnote{\tiny Boukamp, B. A. (1995). A Linear Kronig‐Kramers Transform Test for Immittance Data Validation. Journal of The Electrochemical Society, 142(6), 1885–1894.}
\item Automated algorithm is implemented\footnote{\tiny Schönleber, M., Klotz, D., \& Ivers-Tiffée, E. (2014). A Method for Improving the Robustness of linear Kramers-Kronig Validity Tests. Electrochimica Acta, 131, 20–27.}
\item Simple interface:\\
\lstinline{results, mus, residuals = fitter.linkk_test()}
\item Test: let's check two exemplary files
\end{itemize}
\end{frame}

\begin{frame}{EIS data analysis: fitting the data}
\begin{enumerate}
\item Choose a model 
\item Run the fit
\item If real or imaginary part are close to zero, it is better to check the residual with respect to the absolute value of the impedance
\item If the impedance varies a lot with frequency, it is advised to use a weighting algorithm
\item The fit result should be scrutinised by checking the errors of the variables and their correlations in the fit report
\end{enumerate}
\end{frame}

\begin{frame}{EIS data analysis: post-processing the data}
\begin{itemize}
\item The fit result can be used to compute the impedance of the model and sub-models
\item The different impedance of the model and sub-models can be visually compared
\item The data can be exported straightforwardly
\item If much data has been fitted: ImpedanceFitter can be used to generate histograms  (not shown here, please be referred to online documentation)
\end{itemize}
\end{frame}

\begin{frame}{Applications}
\begin{block}{\tiny Zimmermann, J., et al. (2021) Using a Digital Twin of an Electrical Stimulation Device to Monitor and Control the Electrical Stimulation of Cells in vitro. Front. Bioeng. Biotechnol. 9:765516.}
\begin{itemize}
\item EIS of a stimulation device
\item Extraction of different parameters related to electrical, thermal and (electro-)chemical effects by fitting
\item Computation of impedances as input for models
\end{itemize}
\end{block}
\begin{block}{\tiny Zimmermann, J., \& van Rienen, U. (2021). Ambiguity in the interpretation of the low-frequency dielectric properties of biological tissues. Bioelectrochemistry, 140, 107773.}
\begin{itemize}
\item Comparison of measured data and a parametric model
\item Data analysis and model analysis
\item Suggested correction (see \lstinline{Correction-Ambiguity.ipynb})
\end{itemize}
\end{block}
\end{frame}

\end{document}